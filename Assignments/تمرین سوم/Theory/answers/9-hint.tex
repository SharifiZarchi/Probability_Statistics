\subsection*{الف}
	$T_a \leq n$ به این معناست که فلامینگو در طی مسیر حداقل یک بار از $a$ رد شده است. زمانی که وی به نقطه $a$ میرسد دقیقن بین نقطه $a+c$ و $a-c$ قرار دارد و احتمال ورود به هر یک از این دو نقطه در قدم های باقیمانده برابر است بنابراین $P(S_n = a-c,T_a\leq n) = P(S_n = a+c)$\\
\subsection*{ب}
فلامینگو در ابتدا به احتمال مساوی به چپ یا راست میرود و با توجه به تقارن احتمال بازگشت از منفی یک به صفر با احتمال بازگشت از یک به صفر برابر و همان $P(T_1 = 2k-1)$ میباشد و طبق رابطه گفته شده داریم \\
$$P(T_1 = 2k-1) = \frac{1}{2k-1}P(S_{2k-1} = 1)$$
$$P(S_n = x) = {n \choose \frac{n+x}{2}}*\frac{1}{2^n}$$
$$\implies P(T_0=2k) = 2*\frac{1}{2}P(T_1=2k-1) = \frac{{2k \choose k}}{(2k-1)2^{2k}}$$

