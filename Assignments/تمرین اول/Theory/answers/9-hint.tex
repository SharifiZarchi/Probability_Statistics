در حالت کلی برای انتخاب سه نفر از هفت نفر \(\binom{7}{3}\) راه داریم. حال از این تعداد حالاتی را که حداقل دو نفر از سه نفر انتخاب شده کنار هم باشند را محاسبه می کنیم. برای این کار ابتدا تعداد حالاتی را که هیچ دو نفری کنار هم نیستند را به دست آورده و سپس با بهره گیری از اصل متمم به حل سوال میپردازیم. با کمی دقت درمیابیم که به ازای هر انتخاب برای نفر اول، سه انتخاب برای دو نفر دیگر داریم.از طرفی هر حالت سه بار شمرده می شود.\\
برای انتخاب سه نفر که هیچ یک کنار یکدیگر نباشند 
\(\frac{7 \times 3}{3} = 7\)
حالت داریم.
پس\\ خواهیم داشت : 
 \(\frac{4}{5}\)\) = \(1 - \frac{7}{\(\binom{7}{3}\)} 