\subsection*{الف}
\begin{center}
	$ F_L(L < l) = 1 - F_L(L \geq l) = 1 - (1 - l)^3 $
\end{center}
در نتیجه، داریم:

\begin{center}
	$ f_L(l) = 3(1 - l)^2 $
\end{center}

\subsection*{ب}
پیشامد
$ L \geq l, M \leq m $
معادل با این است که هر سه متغیر در بازه
$ l $
تا
$ m $
قرار گیرند.
\\
همچنین میدانیم:
\begin{center}
	$ P(M < m) = P(M < m, L < l) + P(M < m, L \geq l)$
\end{center}
حال با توجه به اینکه
$ P(M < m) = m^3 $
است، توزیع CDF دو متغیر مذکور به شکل زیر است.
\begin{center}
	$ F_{M,L}(m, l) = m^3 - (m - l)^3 $
\end{center}
در نتیجه:
\begin{center}
	$ f_{M,L}(m, l) = 6(m - l) $
\end{center}
