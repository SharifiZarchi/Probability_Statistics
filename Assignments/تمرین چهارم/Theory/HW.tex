\documentclass[12pt,a4paper]{article}
\usepackage{float}
\usepackage{commons/course}
\usepackage{xepersian}


\hidesolutions

\شروع{نوشتار}

\سربرگ{تمرین 4}{}{توزیع توام، توزیع شرطی و حاشیه‌ای، نامساوی های احتمالاتی}{گردآورندگان: امیررضا کاظمی، روزبه مشکین، پدرام خرسندی، آیدا رمضانی}

\section*{توزیع توام، توزیع شرطی و حاشیه‌ای، نامساوی های احتمالاتی
}


\مسئله{دست گرمی*}

در هر مورد تابع توزیع خواسته شده را به دست آورید و سپس تحقیق کنید متغیر تصادفی مورد نظر از چه خانواده ای از توزیع هاست. و با استفاده از آن امید ریاضی و واریانس توزیع را به دست آورید.
\subsection*{الف}
فرض کنید متغیر تصادفی $X$ از توزیع پارتو با متغیر $\theta > 0$ پیروی می کند که تابع توزیع آن به صورت زیر است:
$$f_{X}(x) = \frac{\theta}{x ^ {\theta + 1}} \quad x > 1$$
حال اگر متغیر تصادفی $Y$ به صورت زیر به دست بیاید، $Y$ از چه توزیعی پیروی می کند؟
$$ Y = \ln X$$

\subsection*{ب}
فرض کنید $Y$ متغیر تصادفی با توزیع نرمال استاندارد باشد. یعنی $Y \sim exponential(\lambda)$. حال اگر متغیر تصادفی $W$ به صورت زیر به دست بیاید، تابع توزیع آن را به دست بیاورید.$W$ از چه توزیعی پیروی می کند؟

$$W = \sqrt{Y}$$
\subsection*{ج}
فرض کنید $Z$ متغیر تصادفی با توزیع نرمال استاندارد باشد یعنی $Z \sim N(0, 1)$ . حال اگر متغیر تصادفی $Y$ به صورت زیر به دست بیاید، تابع توزیع آن را به دست بیاورید.$Y$ از چه توزیعی پیروی می کند؟
$$Z = e ^ Z$$
\subsection*{د}
فرض کنید $Z$ متغیر تصادفی با توزیع نرمال استاندارد باشد یعنی $Z \sim N(0, 1)$ . حال اگر متغیر تصادفی $X$ به صورت زیر به دست بیاید، تابع توزیع آن را به دست بیاورید.$X$ از چه توزیعی پیروی می کند؟
$$X = Z ^ 2 $$


\حل{
	

می دانیم اگر $Y = g(X)$  و ریشه های معادله $y = g(x)$ به صورت $x_1 , x_2, ...$ باشند، داریم:
$$f_{Y}(y) = \sum \frac{f_{X}(x_{i})}{|g'(x)|}$$

\subsection*{الف}
$$Y = \ln(X) \implies X = e^{Y}$$
$$f_{Y}(y) = \frac{\frac{\theta}{x^{\theta + 1}}}{\frac{1}{x}} = \theta x^{-\theta} \implies f_{Y}(y) = \theta e ^ {-y\theta}$$
پس $Y$ از توزیع نمایی با پارامتر $\theta$ است و در نتبجه :
$$ E[y] = \frac{1}{\theta}, Var(y) = \frac{1}{\theta ^ 2} $$
\subsection*{ب}
$$f_{W}(w) = \frac{\lambda e^{-\lambda y}}{\frac{1}{2 \sqrt{y}}}, y = w^{2} \implies 2w\lambda e^{-\lambda w^{2}} \sim weibull(\alpha = 2 , \beta = \frac{1}{\lambda})$$

$$E[W] = \sqrt{\beta} \times \Gamma (1 + \frac{1}{\alpha}) = \frac{1}{\sqrt{\lambda}}\Gamma (\frac{3}{2})$$
$$Var[W] = \beta[\Gamma(1 + \frac{2}{\alpha}) - (\Gamma(1 + \frac{1}{\alpha})) ^ {2}] = \frac{1}{\lambda}[\Gamma(2) - (\Gamma(\frac{3}{2})) ^ {2}]$$
\subsection*{ج}
$$f_{X}(x) = \frac{\frac{1}{\sqrt{2\pi}}e^{\frac{-z^2}{2}}}{e^{z}} = \frac{1}{\sqrt{2\pi}}e^{-\frac{(\ln x) ^ 2}{2}} \sim lognormal(\mu = 0, \sigma ^ 2 = 1)$$
$$E[x] = exp(\mu + \frac{\sigma ^ 2}{2}) = e ^ {\frac{1}{2}}$$
$$Var[x] = [exp(\sigma ^ 2) - 1]exp(2\mu + \sigma ^ 2) = e^2 - e$$
\subsection*{د}
$$x = z ^ 2 \implies z_{1} = \sqrt{x}, z_{2} = -\sqrt{x}$$
$$f_{X}(x) = \sum \frac{\frac{1}{\sqrt{2\pi}}e ^ {\frac{-z_{i} ^ 2}{2}}}{2|z_{i}|} = 2 \times \frac{1}{2\sqrt{2\pi x}}e ^ {\frac{-x}{2}} = \frac{1}{\sqrt{2 \pi x}}x ^ {-\frac{1}{2}}e ^ {-\frac{x}{2}} \sim gamma(k = \frac{1}{2} , \theta = 2)$$
$$E[X] = k\theta  = 1$$
$$Var[X] = k\theta ^ 2 = 2$$
دقت کنید که به طور کلی نیازی به در نظر گرفتن ضرایب ثابت در توزیع ها وجود ندازد ، چرا که نهایتا جمع(انتگرال) تابع توزیع چگالی برابر یک خواهد بود.

}


\مسئله{دست گرمی ۲}

تعداد زیرمجموعه های k عضوی از مجموعه ی $\{1, 2, 3, ..., n\}$ را بیابید که هیچ دو عضوی از آن متوالی نباشند.


\حل{
	\subsection*{الف}
\begin{center}
	$ F_L(L < l) = 1 - F_L(L \geq l) = 1 - (1 - l)^3 $
\end{center}
در نتیجه، داریم:

\begin{center}
	$ f_L(l) = 3(1 - l)^2 $
\end{center}

\subsection*{ب}
پیشامد
$ L \geq l, M \leq m $
معادل با این است که هر سه متغیر در بازه
$ l $
تا
$ m $
قرار گیرند.
\\
همچنین میدانیم:
\begin{center}
	$ P(M < m) = P(M < m, L < l) + P(M < m, L \geq l)$
\end{center}
حال با توجه به اینکه
$ P(M < m) = m^3 $
است، توزیع CDF دو متغیر مذکور به شکل زیر است.
\begin{center}
	$ F_{M,L}(m, l) = m^3 - (m - l)^3 $
\end{center}
در نتیجه:
\begin{center}
	$ f_{M,L}(m, l) = 6(m - l) $
\end{center}

}

\مسئله{متغیر تصادفی های مستقل*}

\subsection*{الف}
فرض کنید در حال تست کردن ۳ لامپ از شرکت های مختلف هستیم. تاریخ انقضای هر کدام از لامپ های $P_1$ ، $P_2$ و $P_3$ متغیر تصادفی های نمایی با امید ریاضی به ترتیب 
$\frac{1}{2}$ ،$\frac{1}{4}$ ،$\frac{1}{8}$
می‌باشد (بر حسب سال). با فرض اینکه تاریخ انقضای لامپ ها مستقل از هم باشند، اگر T متغیر تصادفی مدت زمانی باشد که هر 3 لامپ روشن هستند تابع چگالی احتمال آن را بیابید.  


\subsection*{ب}
متغیر تصادفی $X$  دارای تابع چگالی احتمال $f_X(x)$  می‌باشد. $c$ را به گونه ای بیابید که
 $E_X[\mid X - c \mid ]$
 کمینه شود. 





\حل{																																																				
	

\subsection*{الف}
		$$ 1 - \left ( \frac{5 ^ 0 e ^ {-5}}{0!} + \frac{5 ^ 1 e ^ {-5}}{1!} + \frac{5 ^ 2 e ^ {-5}}{2!} + \frac{5 ^ 3 e ^ {-5}}{3!}\right ) $$
\subsection*{ب}
		$$ 1 - (1 - e ^ {-5 \times 0.5}) = e ^ {-2.5} = 0.082$$
\subsection*{ج}
با توحه به بی حافظه بودن توزیع نمایی ، زمان پنچمین سانحه از لحظه فعلی از توزیع جمع 5 متغیر تصادفی با توزیع نمایی پیروی می کند. بنابراین  از توزیع $Gamma(5 , 5)$ پیروی می کند.کافیست مساحت زیر نمودار توزیع احتمال گفته شده را تا قبل از 0.75 بیابیم که برابر است با:
$$ 5.585808 \times 10 ^ {-7}$$ 



}

\مسئله{توزیع توام}
در کیسه ای 4 توپ قرمز و 5 توپ آبی قرار دارد.
اگر از این کیسه 4 توپ را به صورت تصادفی در بیاوریم چقدر احتمال دارد 2 تای اولی قرمز و 2 تای بعدی آبی باشند؟

\\


\حل{
	\subsection*{الف}

با توجه به اینکه توزیع عدد گزارش شده توسط هر دستگاه* توزیع نرمالی است که واریانس آن را نداریم، می‌توانیم توزیع میانگین را
\lr{Student's t}
در نظر بگیریم. در این صورت برای بازه اطمینان داریم
$$-t_{.025} \leq \frac{\bar{x} - \mu}{\frac{s}{\sqrt{n}}} \leq t_{.025}$$
از روی داده‌ها داریم
$\bar{x} = 151.5 \quad s = 2.38$
و با توجه به اینکه درجه آزادی در اینجا ۳ است از روی جدول
\lr{t}
داریم
$t_{.025} = 3.18$.
با جایگذاری مقادیر بازه اطمینان
$[147.7, 155.3]$
به دست می‌آید.

\subsection*{ب}
از آزمون تک نمونه‌ای
\lr{t-test}
با فرض صفر
$\mu = 150$
و فرض دیگر
$\mu > 150$
استفاده می‌کنیم. 
\lr{t-value}
داده‌ها برابر است با
$\frac{\bar{x} - \mu}{\frac{s}{\sqrt{n}}} = 1.26$
که از
$t_{0.05} = 2.35$
کمتر است بنابراین نمی‌توانیم فرض صفر را رد کنیم.

\subsection*{پ}
خطای نوع اول برابر با سطح اهمیت آزمون است بنابراین برای کاهش خطای نوع اول باید مقدار عددی سطح اهمیت را کاهش دهیم. خطای نوع دوم هنگامی اتفاق می‌افتد که فرض صفر غلط باشد و ما به اشتباه نتوانیم آن را رد کنیم. افزایش مقدار عددی سطح اهمیت رد کردن فرض صفر را ساده کرده و احتمال آن را بیشتر می‌کند (در حالی که تاثیری روی درست یا غلط بودن فرض صفر در عالم واقعیت ندارد!) بنابراین افزایش مقدار عددی سطح اهمیت باعث کاهش احتمال خطای نوع دوم می‌شود.
}

\مسئله{شمارش تخم مرغی}
50 بادکنک در جعبه ای قرار دارند که می دانیم 5 تای آنها سوراخ دارند.
احتمال این که 3 بادکنک برداریم و همگی سالم باشند چقدر است؟

\\


\حل{
	
متغیر تصادفی $X$ را متناظر با عمر چرخ در نظر بگیرید. در نتیجه
$X \sim N(34000,4000^2)$

\subsection*{الف}
$$P(X > 40000) = 1 - P(X \leq 40000) = 1 - P(\frac{X - 34000}{4000} \leq \frac{40000 - 34000}{4000})$$
$$ = 1 - \phi(1.5) = 1 - 0.93319 = 0.06681$$
\subsection*{ب}
$$P(30000 \leq X \leq 35000) = P(\frac{30000 - 34000}{4000} \leq \frac{X - 34000}{4000} \leq \frac{35000 - 34000}{4000})$$
$$ = P(-1 \leq \frac{X - 340000}{4000} \leq 0.25) = \phi(0.25) - \phi(-1) = 0.59871 - 0.15866 = 0.44$$
\subsection*{ج}
$$P(X \geq 40000 \mid X \geq 30000) = \frac{P(X \geq 40000, X \geq 30000)}{P(X \geq 30000)} = \frac{P(X \geq 40000)}{P(X \geq 30000)}$$
$$= \frac{1 - P(\frac{X - 34000}{4000} \leq \frac{40000 - 34000}{4000})}{1 - P(\frac{X - 34000}{4000} \leq \frac{30000 - 34000}{4000})} = \frac{1 - \phi(1.5)}{1 - \phi(-1)}$$
$$ = \frac{1 - 0.93319}{1 - 0.15866} = 0.0794$$


}

%%%%%%

\مسئله{جمع متغیرهای تصادفی*}
دو سکه داریم که با پرتاب اولی به احتمال
$0.5$
شیر و به احتمال
$0.5$
خط میاید 
و با پرتاب دومی به احتمال $0.6$ شیر ظاهر می شود و به احتمال $0.4$ خط.

یک سکه را شانسی برداشته ایم و پس از 3 بار پرتاب کردن آن، به ترتیب به نتایج خط-شیر-شیر دست یافته ایم.

احتمال این که پرتاب چهارم خط بیاید چقدر است؟
\\


\حل{
	ابتدا تابع توزیع احتمال Z را محاسبه می‌کنیم. می‌دانیم تابع توزیع Z برابر است با convolution تابع‎های توزیع X و Y. از طرفی با استفاده از شهود نموداری convolution می‌توانیم تشخیص دهیم که تابع توزیع Z برای 
$ z < 0 $
برابرِِ صفر است. حالا برای محاسبه تابع توزیع برای مقادیر بزرگتر از صفر داریم:

\begin{align*}
	f_Z(z) &= \int_{-\infty}^{\infty} f_X(z - y)f_Y(y) \, dy \\
	&= \int_{0}^{\infty} f_X(z - y)e^{-y} \, dy
\end{align*}
از طرفی می‌دانیم زمانی 
$f_X(z - y)$
برابرِ 1 است که
\begin{align*}
0  \le z - y \le 1
&\equiv \quad -1 \le y-z \le 0 \\ 
&\equiv \quad z - 1 \le y \le z 
\end{align*}
بنابراین داریم:
\begin{align*}
f_Z(z)
&= \int_{max(z - 1, 0)}^{z} e^{-y} \, dy \\
&= -e^{-y} |_{max(0, z-1)}^{z} \\
&= -e^{-z} + e^{-max(0, z-1)} \\
&= 
\begin{cases}
	e^{1 - z} - e^{-z} & \quad z \ge 1 \\
	1 - e^{-z} & \quad z < 1 \\
\end{cases}
\end{align*}


\subsection*{الف}
طبق آنچه گفته شد پاسخ این قسمت برابر است با:
$$
f_Z(z) =
\begin{cases}
	e^{1 - z} - e^{-z} & \quad z \ge 1 \\
	1 - e^{-z} & \quad z < 1 \\
\end{cases}
$$

\subsection*{ب}
برای محاسبه تابع توزیع تجمعی 3 حالت داریم. در حالت
$ z < 0 $
تابع توزیع تجمعی برابرِ صفر است. در حالت
$ 0 \le z < 1 $
:
\begin{align*}
F_Z(z) 
&= \int_{0}^{z} 1 - e^{-w} \, dw \\
&= z + e^{-w}|_{0}^{z} \\
&= z + e^{-z} - 1
\end{align*}
در حالت 
$ 1 \le z $
داریم:
\begin{align*}
F_Z(z)
&= \int_{1}^{z} e^{1 - w} - e^{-w}\, dw \\
&= - e^{1 - w} |_{1}^{z}  + e^{-w} |_{1}^{z} \\
%%&= -e ^ {1-z} + e + e^{-z} - 1
\end{align*}

















































}

\مسئله{جمع متغیرهای تصادفی گسسته}

$X$ را تغداد نقاط ثابت برای یک جایگشت تصادفی از 1 تا n در نظر بگیرید. (نقطه ثابت: $\pi(i)=i$ ) میخواهیم نشان دهیم $x$ غیر ممکن است خیلی بیشتر از یک شود. برای این کار امید ریاضی و واریانس $X$ را به دست آورده و استدلال کنید\\


\حل{
	
$X_i$ را متناظر با ثابت بودن $i$ در نظر میگیریم داریم\\
$$X_i =\begin{cases}
1 &\pi(i) = i\\
0  &o.w
\end{cases}\implies p(X_i = 1) = \frac{1}{n}$$
\\
$$X=\sum_{i=1}^{n}X_i \rightarrow E[x] = \sum_{i=1}^{n}E[X_i] = n * \frac{1}{n} = 1$$
$$E[X^2] = E[\sum X_i^2 + 2\sum_{i<j} X_iX_j]$$
$$E[\sum X_i^2] = E[\sum X_i] = 1$$
$$E[\sum_{i<j} X_iX_j] = \frac{n(n-1)}{2} * (\frac{1}{n}*\frac{1}{n-1}) = 1 \rightarrow E[x^2] = 1 + 2 = 3$$
$$Var(x) = E[x^2] - E[x]^2 = 3 - 1 = 2$$
همان گونه که میبینید امید ریاضی 1 و با توجه به کوچک بودن واریانس احتمال وقوع اعداد بسیار بزرگتر از یک بسیار کم است.
}
%%%%%%



\مسئله{تابع توزیع متغیرهای تصادفی مستقل}
ثابت کنید متغیرهای
$ X_1 $
, ... ,
$ X_n $
مستقلند اگر و تنها اگر تابع توزیع توام آن‌ها به شکل زیر قابل بیان باشد.
\begin{center}
	$f( x_1, x_2, ..., x_n ) = \prod_{i=1}^n g_i(x_i)$
\end{center}
$ g_i $
تابعی مثبت است.

\حل{
	با استفاده از استقرا قابل است که:
\begin{center}
	$ \int_{-\infty}^{\infty} g_i(x_i) dx_i = 1$
\end{center}
.
برای اثبات حکم روی
$ n $
استقرا میزنیم.
\\
پایه استقرا: برای
$ n = 2 $
میدانیم، اگر
$ x_1 $
و
$ x_2 $
مستقل باشند، آنگاه؛
\begin{center}
	$ f(x_1, x_2) = f_1(x_1) f_2(x_2) $
\end{center}
است. در نتیجه توزیع توام این دو متغیر به شکل 
$ g_1(x_1) g_2(x_2) $
قابل بیان است.
\\
همچنین اگر 
$ f(x_1, x_2) = g_1(x_1) g_2(x_2) $
داریم:
\begin{center}
	$ f_2(x_2) = \int_{-\infty}^{\infty} f(x_1, x_2) dx_1 = \int_{-\infty}^{\infty} g_2(x_2) g_1(x_1) dx_1 = g_2(x_2)$
	\\
	$ f_1(x_1) = \int_{-\infty}^{\infty} f(x_1, x_2) dx_2 = \int_{-\infty}^{\infty} g_1(x_1) g_2(x_2) dx_2 = g_1(x_1)$
	\\
	$ \rightarrow f(x_1, x_2) = g_1(x_1) g_2(x_2) = f_1(x_1) f_2(x_2)$
\end{center}
بنابراین دو متغیر 
$ x_1 $
و
$ x_2 $
مستقلند.
گام استقرا:
فرض کنید متغیرهای
$ x_1,..., x_n, x_{n+1} $
از یکدیگر مستقل باشند.
حال با توجه به فرض استقرا؛
\begin{center}
	$ f(x_1,..., x_n | x_{n+1}) = f(x_1,..., x_n) f_{n+1}(x_{n+1}) = \prod_{i=1}^{n} g_i(x_i) f_{n+1}(x_{n+1})$
\end{center}
همانطور که مشاهده شد فرم مذکور در صورت سوال قابل مشاهده است.
\\
حال فرض میکنیم،
$ f(x_1,..., x_n, x_{n+1}) = \prod_{i=1}^{n+1} g_i(x_i) $
است.
\\
با انتگرالگیری روی 
$ x_{n+1} $
داریم:
\begin{center}
	$ f(x_1,..., x_n) = \prod_{i=1}^{n} g_i(x_i) $
\end{center}
باتوجه به فرض استقرا متغیرهای
$ x_1,..., x_n $
از یکدیگر مستقلند. میتوان معادله‌ی بالا را فرم زیر نوشت.
\begin{center}
	$ f(x_1,..., x_n, x_{n+1}) = f(x_1,..., x_n) g_{n+1}(x_{n+1}) $
\end{center}
با انتگرالگیری روی
$ x_1 $
تا
$ x_n $
نتیجه میشود:
\begin{center}
	$ f_{n+1}(x_{n+1}) = \int_{-\infty}^{\infty} ... \int_{-\infty}^{\infty} f(x_1,..., x_n, x_{n+1}) dx_1 ... dx_n = g_{n+1}(x_{n+1}) $
	\\
	$ \rightarrow f(x_1,..., x_n, x_{n+1}) = f(x_1,..., x_n) f_{n+1}(x_{n+1}) $
\end{center}
در نتیجه متغیرهای مستقلند.
}





\مسئله{نامساوی مارکوف}
در یک جعبه m توپ قرمز و n توپ آبی داریم.در هر مرحله یک توپ را به تصادف از جعبه خارج میکنیم تا زمانی که در مجموع r توپ قرمز دیده باشیم. احتمال این که در مجموع k توپ از جعبه بیرون آورده باشیم چه قدر است؟ 


\حل{
	در حالت کلی برای انتخاب سه نفر از هفت نفر \(\binom{7}{3}\) راه داریم. حال از این تعداد حالاتی را که حداقل دو نفر از سه نفر انتخاب شده کنار هم باشند را محاسبه می کنیم. برای این کار ابتدا تعداد حالاتی را که هیچ دو نفری کنار هم نیستند را به دست آورده و سپس با بهره گیری از اصل متمم به حل سوال میپردازیم. با کمی دقت درمیابیم که به ازای هر انتخاب برای نفر اول، سه انتخاب برای دو نفر دیگر داریم.از طرفی هر حالت سه بار شمرده می شود.\\
برای انتخاب سه نفر که هیچ یک کنار یکدیگر نباشند 
\(\frac{7 \times 3}{3} = 7\)
حالت داریم.
پس\\ خواهیم داشت : 
 \(\frac{4}{5}\)\) = \(1 - \frac{7}{\(\binom{7}{3}\)} 
}





\مسئله{نامساوی چبیشف}
فرض کنید متغیر تصادفی X از یک توزیع
$
Binomial(n, p)
$
می‌آید.
\subsection*{الف}
با فرض اینکه 
$
p < \alpha < 1
$
و استفاده از نابرابری چبیشف، یک کران بالا برای 
$
P(X \ge \alpha n)
$
بیابید.

\subsection*{ب}
مقدار کران بالا را برای 
$
p = \frac{1}{2}
$
و
$
\alpha = \frac{3}{4}
$ 
تعیین کنید.


\حل{
	
احتمال سالم ماندن یک هواپیمای ۵ موتوره برابر مقدار زیر است:
$${5 \choose 0} \times p ^ 4  + {5 \choose 1} \times p ^ 3 (1 - p) + {5 \choose 2} \times p ^ 3 (1 - p) ^ 2$$
همچنین احتمال سالم ماندن یک هواپیمای ۳ موتوره برابر مقدار زیر است:
$$ {3 \choose 0} \times p ^ 3 + {3 \choose 1} \times p ^ 2(1 - p) $$
پس $p$ باید مقداری باشد که احتمال سالم ماندن هواپیمای 5 موتوره بیشتر از هواپیما 3 موتوره باشد در نتیجه با حل نامساوی بالا مقدار $p$ به دست می آید:
$$ p > 0.2706$$

}



\مسئله{تراشه ها*}
حدود 2 درصد از تراشه‌های RAM تولید شده در یک کارخانه خراب است. علی به 50 عدد RAM سالم برای آزمایشگاهش نیاز دارد. علی باید چه تعداد تراشه RAM خریداری کند تا مطمئن باشد با احتمال حداقل 99 درصد، حداقل 49 تراشه سالم RAM دارد؟


\حل{
	طبق نابرابری مارکوف داریم:
$$
P(X \ge 49) \le \frac{E(X)}{49}
$$
از طرفی،
$$
P(X \ge 49) \ge \frac{99}{100}
$$
پس
\begin{align*}
&\frac{E(X)}{49} \ge \frac{99}{100}\\
&\Rightarrow \frac{N\frac{98}{100}}{49} \ge \frac{99}{100} \\
&\Rightarrow \frac{2N}{100} \ge \frac{99}{100} \\
& \Rightarrow N \ge 50
\end{align*}

بنابراین حداقل 50 قطعه RAM باید خریداری شود.

}

\begin{flushleft}
	موفق باشید
\end{flushleft}



\پایان{نوشتار}
