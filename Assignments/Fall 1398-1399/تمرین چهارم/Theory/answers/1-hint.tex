

\subsection*{الف}
راه حل بخش اول \\ 
از آنجا که متغیر تصادفی های $P_1$، $P_2$ و $P_3$ مستقل و همگی دارای توزیع نمایی هستند بنابراین خواهیم داشت: 
$$
P_1 = exp(2) , \,   P_2 = exp(4) , \,   P_3 = exp(8)
$$
 متغیر تصادفی $T$ مدت زمانی است که هر 3 لامپ روشن هستند بنابراین میتوان $T$ را بر اساس   $P_1 $ ، $P_2 $ و $P_3 $ نوشت .
$$
T = min(P_1, P_2, P_3)
$$
 حال برای محاسبه تابع توزیع چگالی از تابع توزیع تجمعی کمک میگیریم که با تعاریف بالا خواهیم داشت:
$$ 
P(T \ge t) = P( P_1 \geq t , P_2 \geq t, P_3 \geq t)
$$
$$
P(T \geq t) = P( P_1 \geq t) P( P_2 \geq t) P( P_3 \geq t)  
$$
به صورت کلی برای توزیع توانی داریم: 
$$
P(X \geq x) = 1 - P(X \leq x) = 1 - F_X(x) = 1 - \int_{0}^{x} \lambda e^{-\lambda x} \, dx = e^{-\lambda x}
$$
در نتیجه 
$  P(T \geq t) = e^{-2t}e^{-4t}e^{-8t} = e ^{-14t}
$
 و 
$
F_T(t) = 1 - e ^ {-14t}
$ 
و تابع توزیع چگالی یعنی همان $f_T(t) $ برابر می شود با:
$$ f_T(t) = \frac{\partial F}{\partial t} = 14e^{-14t} .
$$
\subsection*{ب}
راه حل بخش دوم \\
$$
E[\mid X-c \mid ] =\int_{-\infty}^ {\infty} \mid X-c \mid f_X(x) \, dx =\int_{c}^{\infty} (X-c)f_X(x) \, dx +\int_{-\infty}^{c} (c - X)f_X(x) \, dx = g(c)
$$
$$
 \frac{\partial g}{\partial c} = 0 \implies \int_{c}^{\infty}(−1)f_X(x) \, dx + \int_{-\infty}^{c}(1)f(X) \, dx = 0 \implies \int_{c}^{\infty} f_X(x)=\int_{-\infty}^{c}f_X(x)
$$
$$
 \implies c = median(f(X))
$$
