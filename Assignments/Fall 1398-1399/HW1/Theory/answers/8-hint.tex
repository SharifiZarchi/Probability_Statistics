برای این که بعد از دیدن r امین توپ قرمز در کل k توپ دیده باشیم باید در \(k - 1\) انتخاب اول دقیقا \(r - 1\) توپ قرمز و \(k - r \) توپ آبی دیده باشیم و انتخاب آخر نیز قرمز باشد. 
احتمال این که در \(k - 1\) انتخاب اول در مجموع \(r - 1\) توپ قرمز دیده باشیم برابر است با :  \(\frac{\binom{m}{r - 1}\times \binom{n}{k - r}}{\binom{m + n}{k - 1}}\)  
\\
حال با شرط این که در \(k - 1\) انتخاب اول دقیقا\( r - 1\) توپ قرمز دیده باشیم ، احتمال اینکه انتخاب آخر قرمز باشد برابر است\\ با :
 \(\frac{m - r + 1}{m + n - k + 1}\)
\\ \\
پس داریم:\\
احتمال خواسته شده برابر است با ضرب دو عبارت به دست آمده. 
