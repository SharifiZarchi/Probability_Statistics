در این راه‌حل سعی می‌کنیم روند کلی حل این مسئله را توضیح دهیم. برای جزئیات دقیق‌تر به منابع مراجعه کنید.
اگر به بازی خوب نگاه کنیم، می‌توانیم یک خاصیت بازگشتی در آن مشاهده کنیم. 
قدم اول را در نظر بگیرید. اگر الکسی ببازد ، انگار با همان مسئله روبروییم در حالی که سرمایه الکسی یک واحد کمتر باشد و در نتیجه سرمایه رقیب او یک واحد بیشتر. به همین ترتیب برای برد الکسی نیز می‌توان چنین حرفی زد.
(
به‌طور کلی این فرآیند یک 
\lr{Random Walk}
نام دارد که در درس فرآیندهای تصادفی بررسی می‌شود.
).
\\
حال
$P_i$
را بدین صورت تعریف می‌کنیم: احتمال اینکه الکسی بازی را ببرد اگر با 
i
واحد سرمایه بازی را شروع کند.  
سپس با توجه به قانون جمع کل و دید بازگشتی توضیح داده‌شده می‌توان نوشت::

\begin{align*}
    P_i &= p P_{i+1} + q P_{i-1} \text{, } & &1 < i < N-1 \\
    & & & P_0 = 0 \\
    & & & P_N = 1 \\
\end{align*}

معادله‌ی بدست آمده، یک‌جور معادله دیفرانسیل اما در حالت گسسته است. با حل این معادله به جواب زیر برای 
$P_i$
خواهیم رسید.

\begin{align*}
    P_i &=
    \begin{cases}
            \frac{1-\left(\frac{q}{p}\right)^i}{1-\left(\frac{q}{p}\right)^N}  & \quad \text{ if } p \neq q \\
            \frac{i}{N}  & \quad \text{ if } p = q \\
        \end{cases}
\end{align*}

حال با توجه به رابطه بدست‌آمده برای 
$P_i$
می‌توان گزاره‌های موجود در صورت سوال را نتیجه گرفت. به عنوان مثال اگر 
$p = 0.49$
باشد، یعنی بازی خیلی کم غیرعادلانه باشد و حتی سرمایه دو فرد در ابتدا با هم برابر باشد
(
$i = N-i$
و
فرض کنید
$N=200$
)
،
احتمال برنده‌شدن الکسی 
$0.02$
است!
\\
\textbf{کنجکاوی}
:
آیا می‌توان نشان داد که این بازی حتما خاتمه می‌یابد و تا ابد ادامه پیدا نمی‌کند؟