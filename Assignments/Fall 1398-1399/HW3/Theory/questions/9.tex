
فرض کنید یک فلامینگو قابیلت راه رفتن روی اعداد صحیح را دارد.ابتدا روی صفر ایستاده است و در مرحله اول به صورت کاملا تصادفی به سمت راست یا چپ گامی به طول  بر می دارد(با احتمال مساوی) و در مراحل بعدی نیز همین طور عمل می کند. حالت فلامینگو را در مرحله \lr{n} ام با $S_{n}$ نشان می دهیم.

\subsection*{الف}
کمترین تعداد مراحل برای اولین ورود به نقطه $a$ را با $T_{a} = min(n \geq 1 : S_{n} = a)$ نشان می دهیم.نشان دهید برابر هر $a , c$ عضو اعداد طبیعی رابطه زیر درست است.
$$P(S_{n} = a - c , T_{a} \leq n) = P(S_{n} = a + c)$$
\subsection*{ب}
احتمال \lr{$P(T_{0} = 2k)$} به ازای هر $k$ را به دست آورید.

