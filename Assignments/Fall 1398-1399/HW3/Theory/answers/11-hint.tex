\subsection*{الف}
$T_1$ و $T_2$ را به ترتیب زمان ملاقات دانشجوی اول و دوم در نظر میگیریم و زمان بین ورود دانشجوی اول و خروج دانشجوی دوم از رابطه $T = max(T_1,5) + T_2$ به دست می آید. مسئله را به دو حالت $T_1 < 5$ و $T_2 \geq 5$ تقسیم کرده و امید ریاضی $T$ به صورت $E[T] = pr(T_1 < 5)E[T|T_1 < 5] + pr(T_1 \geq 5)E[T|T_1 \geq 5]$ محاسبه میشود.\\

$$pr(T_1 < 5) = F(5) = 1 - e^{-\frac{5}{30}}$$

$$pr(T_1 \geq 5) = 1 - F(5) = e^{-\frac{5}{30}}$$

$$E[T|T_1 < 5] = E[T_2 + 5] = E[T_2] + 5 = 35$$

$$E[T|T_1 \geq 5] = E[T_2 + T_1 | T_1 \geq 5] = E[T_2] +  E[T_1| T_1 \geq 5]$$
 از آنجا که توزیع نمایی یک توزیع بی حافظه است این دانش اولیه که $T_1 \geq 5$ میباشد تاثیری بر مدت زمان باقیمانده از ملاقات ندارد و مدت زمان باقی مانده از همان توزیع نمایی امید ریاضی 30 پیروی میکند. بنابراین $E[T_1| T_1 \geq 5] = 30 +5 = 35$ و بنابراین\\
$$E[T|T_1 \geq 5] = 30 + 35 = 65$$
و با استفاده از رابطه $E[T]$ داریم \\
$$E[T] = (1 - e^{-\frac{5}{30}}) * 35 +  (e^{-\frac{5}{30}}) * 65 \approx 60.4 $$

\subsection*{ب}
متغیر تصادفی $X$ را زمان تاخیر دانشجوی دوم در نظر میگیریم. مانند قسمت قبل به ازای یک $x$ مشخص $E[T] = pr(T_1 < x)E[T|T_1 < x] + pr(T_1 \geq x)E[T|T_1 \geq x]$ و\\ 
$$pr(T_1 < x) = F(x) = 1 - e^{-\frac{x}{30}}$$
$$pr(T_1 \geq x) = 1 - F(x) = e^{-\frac{x}{30}}$$
$$E[T|T_1 < x] = E[T_2 + x] = E[T_2] + x = 30 + x$$
$$E[T|T_1 \geq 5] = E[T_2 + T_1 | T_1 \geq 5] = E[T_2] +  E[T_1] + x = 60 + x$$
$$\rightarrow E[T] = (1 - e^{-\frac{x}{30}}) * (30+x) +  (e^{-\frac{x}{30}}) * (60+x) = 30e^{-\frac{x}{30}} + 30 + x$$
با محاسبه این مقدار برای تمامی $x$ها به ازای احتمال آن ها امید ریاضی خواسته شده به دست می آید\\ 
$$\int_0^{+\infty} (30e^{-\frac{x}{30}} + 30 + x)\frac{1}{5}e^{-\frac{x}{5}}dx \approx 60.7 $$
