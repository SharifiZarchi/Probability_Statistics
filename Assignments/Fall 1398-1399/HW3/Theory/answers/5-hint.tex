
$X_i$ را متناظر با ثابت بودن $i$ در نظر میگیریم داریم\\
$$X_i =\begin{cases}
1 &\pi(i) = i\\
0  &o.w
\end{cases}\implies p(X_i = 1) = \frac{1}{n}$$
\\
$$X=\sum_{i=1}^{n}X_i \rightarrow E[x] = \sum_{i=1}^{n}E[X_i] = n * \frac{1}{n} = 1$$
$$E[X^2] = E[\sum X_i^2 + 2\sum_{i<j} X_iX_j]$$
$$E[\sum X_i^2] = E[\sum X_i] = 1$$
$$E[\sum_{i<j} X_iX_j] = \frac{n(n-1)}{2} * (\frac{1}{n}*\frac{1}{n-1}) = 1 \rightarrow E[x^2] = 1 + 2 = 3$$
$$Var(x) = E[x^2] - E[x]^2 = 3 - 1 = 2$$
همان گونه که میبینید امید ریاضی 1 و با توجه به کوچک بودن واریانس احتمال وقوع اعداد بسیار بزرگتر از یک بسیار کم است.