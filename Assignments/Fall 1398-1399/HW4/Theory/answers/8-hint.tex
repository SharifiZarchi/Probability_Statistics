با استفاده از استقرا قابل است که:
\begin{center}
	$ \int_{-\infty}^{\infty} g_i(x_i) dx_i = 1$
\end{center}
.
برای اثبات حکم روی
$ n $
استقرا میزنیم.
\\
پایه استقرا: برای
$ n = 2 $
میدانیم، اگر
$ x_1 $
و
$ x_2 $
مستقل باشند، آنگاه؛
\begin{center}
	$ f(x_1, x_2) = f_1(x_1) f_2(x_2) $
\end{center}
است. در نتیجه توزیع توام این دو متغیر به شکل 
$ g_1(x_1) g_2(x_2) $
قابل بیان است.
\\
همچنین اگر 
$ f(x_1, x_2) = g_1(x_1) g_2(x_2) $
داریم:
\begin{center}
	$ f_2(x_2) = \int_{-\infty}^{\infty} f(x_1, x_2) dx_1 = \int_{-\infty}^{\infty} g_2(x_2) g_1(x_1) dx_1 = g_2(x_2)$
	\\
	$ f_1(x_1) = \int_{-\infty}^{\infty} f(x_1, x_2) dx_2 = \int_{-\infty}^{\infty} g_1(x_1) g_2(x_2) dx_2 = g_1(x_1)$
	\\
	$ \rightarrow f(x_1, x_2) = g_1(x_1) g_2(x_2) = f_1(x_1) f_2(x_2)$
\end{center}
بنابراین دو متغیر 
$ x_1 $
و
$ x_2 $
مستقلند.
گام استقرا:
فرض کنید متغیرهای
$ x_1,..., x_n, x_{n+1} $
از یکدیگر مستقل باشند.
حال با توجه به فرض استقرا؛
\begin{center}
	$ f(x_1,..., x_n | x_{n+1}) = f(x_1,..., x_n) f_{n+1}(x_{n+1}) = \prod_{i=1}^{n} g_i(x_i) f_{n+1}(x_{n+1})$
\end{center}
همانطور که مشاهده شد فرم مذکور در صورت سوال قابل مشاهده است.
\\
حال فرض میکنیم،
$ f(x_1,..., x_n, x_{n+1}) = \prod_{i=1}^{n+1} g_i(x_i) $
است.
\\
با انتگرالگیری روی 
$ x_{n+1} $
داریم:
\begin{center}
	$ f(x_1,..., x_n) = \prod_{i=1}^{n} g_i(x_i) $
\end{center}
باتوجه به فرض استقرا متغیرهای
$ x_1,..., x_n $
از یکدیگر مستقلند. میتوان معادله‌ی بالا را فرم زیر نوشت.
\begin{center}
	$ f(x_1,..., x_n, x_{n+1}) = f(x_1,..., x_n) g_{n+1}(x_{n+1}) $
\end{center}
با انتگرالگیری روی
$ x_1 $
تا
$ x_n $
نتیجه میشود:
\begin{center}
	$ f_{n+1}(x_{n+1}) = \int_{-\infty}^{\infty} ... \int_{-\infty}^{\infty} f(x_1,..., x_n, x_{n+1}) dx_1 ... dx_n = g_{n+1}(x_{n+1}) $
	\\
	$ \rightarrow f(x_1,..., x_n, x_{n+1}) = f(x_1,..., x_n) f_{n+1}(x_{n+1}) $
\end{center}
در نتیجه متغیرهای مستقلند.