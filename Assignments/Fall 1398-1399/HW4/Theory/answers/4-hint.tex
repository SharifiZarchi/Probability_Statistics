ابتدا تابع توزیع احتمال Z را محاسبه می‌کنیم. می‌دانیم تابع توزیع Z برابر است با convolution تابع‎های توزیع X و Y. از طرفی با استفاده از شهود نموداری convolution می‌توانیم تشخیص دهیم که تابع توزیع Z برای 
$ z < 0 $
برابرِِ صفر است. حالا برای محاسبه تابع توزیع برای مقادیر بزرگتر از صفر داریم:

\begin{align*}
	f_Z(z) &= \int_{-\infty}^{\infty} f_X(z - y)f_Y(y) \, dy \\
	&= \int_{0}^{\infty} f_X(z - y)e^{-y} \, dy
\end{align*}
از طرفی می‌دانیم زمانی 
$f_X(z - y)$
برابرِ 1 است که
\begin{align*}
0  \le z - y \le 1
&\equiv \quad -1 \le y-z \le 0 \\ 
&\equiv \quad z - 1 \le y \le z 
\end{align*}
بنابراین داریم:
\begin{align*}
f_Z(z)
&= \int_{max(z - 1, 0)}^{z} e^{-y} \, dy \\
&= -e^{-y} |_{max(0, z-1)}^{z} \\
&= -e^{-z} + e^{-max(0, z-1)} \\
&= 
\begin{cases}
	e^{1 - z} - e^{-z} & \quad z \ge 1 \\
	1 - e^{-z} & \quad z < 1 \\
\end{cases}
\end{align*}


\subsection*{الف}
طبق آنچه گفته شد پاسخ این قسمت برابر است با:
$$
f_Z(z) =
\begin{cases}
	e^{1 - z} - e^{-z} & \quad z \ge 1 \\
	1 - e^{-z} & \quad z < 1 \\
\end{cases}
$$

\subsection*{ب}
برای محاسبه تابع توزیع تجمعی 3 حالت داریم. در حالت
$ z < 0 $
تابع توزیع تجمعی برابرِ صفر است. در حالت
$ 0 \le z < 1 $
:
\begin{align*}
F_Z(z) 
&= \int_{0}^{z} 1 - e^{-w} \, dw \\
&= z + e^{-w}|_{0}^{z} \\
&= z + e^{-z} - 1
\end{align*}
در حالت 
$ 1 \le z $
داریم:
\begin{align*}
F_Z(z)
&= \int_{1}^{z} e^{1 - w} - e^{-w}\, dw \\
&= - e^{1 - w} |_{1}^{z}  + e^{-w} |_{1}^{z} \\
%%&= -e ^ {1-z} + e + e^{-z} - 1
\end{align*}
















































