\subsection*{الف}

با توجه به اینکه توزیع عدد گزارش شده توسط هر دستگاه* توزیع نرمالی است که واریانس آن را نداریم، می‌توانیم توزیع میانگین را
\lr{Student's t}
در نظر بگیریم. در این صورت برای بازه اطمینان داریم
$$-t_{.025} \leq \frac{\bar{x} - \mu}{\frac{s}{\sqrt{n}}} \leq t_{.025}$$
از روی داده‌ها داریم
$\bar{x} = 151.5 \quad s = 2.38$
و با توجه به اینکه درجه آزادی در اینجا ۳ است از روی جدول
\lr{t}
داریم
$t_{.025} = 3.18$.
با جایگذاری مقادیر بازه اطمینان
$[147.7, 155.3]$
به دست می‌آید.

\subsection*{ب}
از آزمون تک نمونه‌ای
\lr{t-test}
با فرض صفر
$\mu = 150$
و فرض دیگر
$\mu > 150$
استفاده می‌کنیم. 
\lr{t-value}
داده‌ها برابر است با
$\frac{\bar{x} - \mu}{\frac{s}{\sqrt{n}}} = 1.26$
که از
$t_{0.05} = 2.35$
کمتر است بنابراین نمی‌توانیم فرض صفر را رد کنیم.

\subsection*{پ}
خطای نوع اول برابر با سطح اهمیت آزمون است بنابراین برای کاهش خطای نوع اول باید مقدار عددی سطح اهمیت را کاهش دهیم. خطای نوع دوم هنگامی اتفاق می‌افتد که فرض صفر غلط باشد و ما به اشتباه نتوانیم آن را رد کنیم. افزایش مقدار عددی سطح اهمیت رد کردن فرض صفر را ساده کرده و احتمال آن را بیشتر می‌کند (در حالی که تاثیری روی درست یا غلط بودن فرض صفر در عالم واقعیت ندارد!) بنابراین افزایش مقدار عددی سطح اهمیت باعث کاهش احتمال خطای نوع دوم می‌شود.